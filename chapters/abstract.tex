
\begin{abstract}

随着数码相机以及移动设备的普及,图像以其直观性和丰富的信息承载量,已经逐渐成为人们记录日常生活的主要方式之一。对图像内容进行自动处理和分析,从而提取出特定的、有价值的信息,也成为一个亟待解决的问题。然而,对图像内容的处理和分析存在诸多难点:一方面,现有的图像信息的存储均是基于离散的像素表示,缺乏对图像语义信息的描述,计算机很难通过离散的像素信息推测出其代表的语义信息;另一方面,随着图像数量的爆炸式增长,计算机需要在有限的硬件环境下,在人们可以容忍的延迟内检索出需要的内容信息,这对图像处理算法提出了十分巨大的挑战。如何使计算机在一定程度上理解图像内容,并更好更快的分析、整理出对人们有用的信息,是当前计算机视觉与多媒体内容分析领域的重要研究课题。

图像的显著性区域检测为解决上述问题提供了一个方向。首先,通过对图像进行显著性分析,可以将图像区域进行更深层次的加工与聚类,实质上是提供了一定的语义级信息,有利于计算机对图像的理解;其次,在得到显著性区域的同时,排除了大部分无关背景信息,可以大大减少计算量。

本文围绕图像的显著性分析及其应用展开了研究,主要成果包括:
\begin{enumerate}
  \item something
  \item something
  \item something
\end{enumerate}

\keywords{计算机视觉,\quad{}显著性区域检测,\quad{}图像处理,\quad{}目标检测,\quad{}人脸识别,\quad{}图像检索}

\end{abstract}

\begin{englishabstract}

This paper is an introduction to \LaTeX{} document class
\texttt{ICTthesis}. A brief guideline for writing thesis is
also included.

\englishkeywords{Institute of Computing Technology, Chinese
Academy of Sciences \mbox{(ICT, CAS)},\quad{}Thesis,
\quad\LaTeX, ...}

\end{englishabstract}
