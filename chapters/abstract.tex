
\begin{abstract}

随着数码相机以及移动设备的普及,图像已经逐渐成为人们记录日常生活的主要方式之一。对图像内容进行自动处理和分析,从而提取出特定的、有价值的信息,也成为一个亟待解决的问题。然而,对图像中一般物体的定位与检测一直是一个难点。

近年来受到广大学者关注的显著区域检测算法为解决上述问题提供了一个方向,通过对图像进行显著性分析,可以检测并定位人们感兴趣的物体,从而为一般物体的检测定位提供了极大的参考价值。本文围绕图像的显著性分析及其应用展开了研究,主要成果包括:
\begin{enumerate}
  \item 深度调研了显著区域检测的各类算法。我们深入研究了显著区域检测的各类典型算法,包括基于局部对比度、基于全局对比度、基于频域分析,以及基于机器学习的算法,同时对比了各类算法的优缺点,为开发性能更好的新算法提供了努力方向。
  \item 基于条件随机场的显著区域检测算法。我们首先提出了全局颜色对比度、全局颜色紧致度、全局颜色中心度以及局部颜色对比度四种显著性特征,随后通过条件随机场,我们有机互补的结合这几种特征,计算图像的显著图。实验表明,我们的方案能够充分的利用各类特征挖掘图像的显著性,在ASD数据集上取得了良好的效果。
  \item 基于蒙特卡洛采样的显著区域检测算法。我们观察到,图像的显著区域在空间上的约束条件能够非常有效的优化检测效果,然而,这些空间约束(紧密性、连通性、包络性)都是二值化的概念,在现有的算法框架下无法实现。为此,我们提出了基于蒙特卡洛采样的显著区域检测框架,并加入了这些空间约束条件。我们在ASD,ECSSD两个公共数据集上进行了实验,实验表明,我们的算法在准确率,召回率上均能达到与经典算法相匹敌的程度,同时拥有较小的时间复杂度。
  \item 显著区域检测在图像检索上的应用。脱离实际应用的算法将失去其价值,为此我们选取了图像检索这个应用场景,验证显著区域检测算法的实际应用价值。我们实现了一个基于经典的BOF模型的图像检索系统,并在该系统上结合了显著区域检测算法。实验表明,显著区域检测算法能有效提升图像检索的性能。
\end{enumerate}

\keywords{计算机视觉,\quad{}图像处理,\quad{}显著性区域检测,\quad{}目标检测,\quad{}图像检索}

\end{abstract}

\begin{englishabstract}

As the population of digital camera and mobile devices, images have already become the frequently used method for people to record their lives. Automatically processing and analysing the images have also been the most wanted problems to be solved. However, general object detection which is many tasks' foundation is something hard to solve until now.

Recently, salient region detection has attracted many researchers' attention, and it provides an solution for general object detection. We made a deep research on this topic, and our mainly achievements are listed as follows:
\begin{enumerate}
  \item A survey about salient region detection has been made. We investigate the state-of-the-art method on salient region detection until now, including local contrast based method, global contrast based method, frequency domain based method and machine learning based method.
  \item Salient region detection based on conditional random field(CRF). We firstly propose some salient cues such as color contrast, color centering, color compactness etc. Then a CRF model is proposed to integrate these salient cues. Experiments show that our method is efficient and practical.
  \item Salient region detection based on Monte Carlo sampling. Based on the observation that salient regions tend to be compact, connected and surrounded, However, concepts of spatial structure only have definite meanings in binary images. Thus, a Monte Carlo Sampling based Saliency model is proposed to utilize these features. Experimental results on two datasets show that, compared with eleven state-of-the-art methods, our approach has a competitive performance and also runs very fast.
  \item Salient region detection applied to image retrieval. We evaluate the performance of salient region detection when applied to image retrieval. A BOF based image retrieval system is realized. Then we integrate salient region detction into this system. Experiments show that, salient region detection indeed raise the performance of image retrieval system.
\end{enumerate}

\englishkeywords{Computer Vision, Image Processing, Salient Region Detection, Object Detection, Image Retrieval}

\end{englishabstract}
