\chapter{总结与展望}
\section{研究工作总结}
随着互联网和多媒体技术的普及,图像的显著性区域检测逐渐成为一个研究热点。在最初的阶段,图像的显著区域检测主要基于局部对比度的计算,通过挖掘图像块的局部稀缺性来标明显著区域。然而,这种方法很容易高亮图像的边缘而并非整个显著性区域。为了克服这些缺陷,研究者又发明了基于全局对比度的计算方法。然而,由于全局对比度的计算复杂度极高,如果在整个颜色空间上做,在现有的硬件资源条件下,计算时间将不可忍受,因而在最初的实现中,只仅仅实现了在亮度分量上的计算。随后,Cheng等人\cite{cheng2011global}发明了颜色量化与色彩空间平滑两个工具,从而将其扩展到了整个颜色空间上。除以上两种方法,还有基于频域的计算,通过将图像转化到频域空间计算其中区块的稀有性。这种方法具有良好的理论基础,然而被研究者们证明其相当于基于局部对比度的方法加上一个高斯平滑,因而与基于局部对比度的方法有一样的缺陷。另外,随着机器学习的火热,近年来也出现了很多将机器学习应用于显著区域检测的方法,但是这些方法训练复杂度高,而且其检测效果常常具有bias,即在不同数据集上的效果相差较大。

为了进一步提升显著区域检测的效果,我们从两方面展开了研究。一方面,我们探索了显著性特征,即什么样的图像被认为是显著的,在此基础上,我们提出了多个显著性特征,并应用在我们的方法中。另一方面,我们研究了如何进行特征的融合,即怎样结合这些特征,产生互补的效果,从而提升算法的性能。在这两个思路的引导下,我们开发了两个算法:基于条件随机场的显著区域检测,以及基于蒙特卡洛采样的显著区域检测。

最后,我们探索了显著区域检测的应用。我们选取了图像检索这个任务,并将显著区域检测算法应用在图像检索中。通过实验对比,我们验证了显著区域检测算法对图像检索效果的提升。

\section{未来工作的展望}
目前显著区域检测在简单背景的情况下工作良好,然而当背景复杂时,仅仅通过像素级的底层信息,还难以达到很好的检测效果。因此在未来的工作中,我们将考虑加入适度的高层次特征,比如人脸识别算子,车牌识别算子等等。另外,目前的显著区域检测算法大多基于单目标物体的先验,当图像中存在多个前景物体时,通常检测效果较差,因此后期的工作也会适度放在解决多目标的情况。

除了显著区域检测算法本身,我们还需将注意力放在应用上。脱离了实际应用的算法,将失去其实际价值,所以我们将从两个方面开展研究:其一,应用的深度问题,如何将显著区域检测与应用深度结合,而并非简单的进行预处理,从而更好的发挥显著区域检测的效果;其二,应用的广度问题,除了本文已探索的图像检索领域,其它领域是否也可以尝试加入显著区域检测以提升效果呢(譬如图像分割,目标识别等等)。

总之,显著区域检测还有很多方面等待改进,同时其应用场景的挖掘也极其具有吸引力和挑战性。未来我们将从算法改进和应用两方面着手继续进行研究工作。
