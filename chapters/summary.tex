\chapter{总结与展望}
\section{研究工作总结}
随着互联网和多媒体技术的普及,图像的显著性区域检测逐渐成为一个研究热点。在最初的阶段,图像的显著区域检测主要基于局部对比度的计算,通过挖掘图像块的局部稀缺性来标明显著区域。然而,这种方法很容易高亮图像的边缘而并非整个显著性区域。为了克服这些缺陷,研究者又发明了基于全局对比度的计算方法。然而,由于全局对比度的计算复杂度极高,如果在整个颜色空间上做,在现有的硬件资源条件下,计算时间将不可忍受,因而在最初的实现中,只仅仅实现了在亮度分量上的计算。随后,Cheng等人\cite{cheng2011global}发明了颜色量化与色彩空间平滑两个工具,从而将其扩展到了整个颜色空间上。除以上两种方法,还有基于频域的计算,通过将图像转化到频域空间计算其中区块的稀有性。这种方法具有良好的理论基础,然而被研究者们证明其相当于基于局部对比度的方法加上一个高斯平滑,因而与基于局部对比度的方法有一样的缺陷。另外,随着机器学习的火热,近年来也出现了很多将机器学习应用于显著区域检测的方法,但是这些方法训练复杂度高,而且其检测效果常常具有bias,即在不同数据集上的效果相差较大。

为了进一步提升显著区域检测的效果,我们从两方面展开了研究。一方面,我们调研了已有的显著性特征,即什么样的图像被认为是显著的,在此基础上,我们提出了多个显著性特征,并应用在我们的方法中。另一方面,我们研究了如何进行特征的融合,即怎样结合这些特征,产生互补的效果,从而提升算法的性能。在这两个思路的引导下,我们研究设计了两套算法:基于条件随机场的显著区域检测,以及基于蒙特卡洛采样的显著区域检测。在基于条件随机场的显著区域检测方法中,我们首先定义了四种显著性特征(全局颜色对比度、全局颜色紧致度、全局颜色中心度、局部颜色对比度),然后通过条件随机场,有机的融合上述特征,产生最终的显著图。在基于条件随机场的显著区域检测方法中,我们挖掘了显著区域的三种空间约束特性(连通性、包络性,紧致性),并利用蒙特卡洛采样,有效的融合了这三种空间特征。实验证明,我们的两种方法均具有较高的检测精度和时间效率。

最后,为了验证算法在典型多媒体内容检索中应用性能,我们选择图像检索任务进行了研究。首先我们分析了现有检索系统存在的不足,然后通过将显著区域检测算法集成到现有检索系统框架中,实现了图像检索性能的提升。

\section{未来工作的展望}
目前显著区域检测在简单背景的情况下工作良好,然而当背景复杂时,仅仅通过像素级的底层信息,还难以达到很好的检测效果。因此在未来的工作中,我们将考虑加入适度的高层次特征,比如人脸识别算子,车牌识别算子等等。另外,目前的显著区域检测算法大多基于单目标物体的先验,当图像中存在多个前景物体时,通常检测效果较差,因此后期的工作也会适度放在解决多目标的情况。

除了显著区域检测算法本身,我们还需将注意力放在应用上。脱离了实际应用的算法,将失去其实际价值。所以未来我们还将考虑其它更多的应用场景,如目标检测、人脸识别等等。

